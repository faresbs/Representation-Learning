% (meta)
% Exercise contributed by Philippe Lacaille 
% label: ch9

\Exercise{
\label{ex:template}
Assume we are given data of size $3\times64\times64$. 
In what follows, provide the correct configuration of a convolutional neural network layer that satisfies the specified assumption. 
Answer with the window size of kernel ($k$), stride ($s$), padding ($p$), and dilation ($d$, with convention $d=1$ for no dilation).
Use square windows only (e.g. same $k$ for both width and height). 
\begin{enumerate}
\item The output shape of the first layer is $(64,32,32)$. 
\begin{enumerate}
    \item Assume $k=8$ without dilation.
    \item Assume $d=7$, and $s=2$.
\end{enumerate}
\item The output shape of the second layer is $(64,8,8)$. 
Assume $p=0$ and $d=1$.
\begin{enumerate}
    \item Specify $k$ and $s$ for pooling with non-overlapping window.
    \item What is output shape if $k=8$ and $s=4$ instead?
\end{enumerate} 
\item The output shape of the last layer is $(128,4,4)$. 
\begin{enumerate}
    \item Assume we are not using padding or dilation.
    \item Assume $d=2$, $p=2$.
    \item Assume $p=1$, $d=1$. 
\end{enumerate}

\end{enumerate}
}

\Answer{This problem was originally solved with the stated requirement to use $d=0$ for no dilation is used. In this case, the relationship between the effective kernel $K$ and the original kernel $k$ became $K=d+k(d+1)$.\\ 
Moving on to the corrected definition of effective kernel when $d=1$ means no dilation, we have:
$K=k+(k-1)(d-1)$.\\ 
The formula for the size of the output is now $o=\frac{i+2p-K}{s}+1$, where $o$ is the size of the output layer after convolution (assuming that the layers are quadratic), $i$ is the input layer size, $p$ is the padding number, $K$ is the size of the effective kernel, and $s$ is the stride. \\

8.1(a): Using the above formula $32=\frac{64+2p-8}{s}+1$, we find that one solution is:\\ ${k=8, s=2, p=3 , d=1}$\\

8.1(b): Given the values for dilation $d=7$ and stride $s=2$, we first calculate the new kernel size  $K=7k-6$. Inserted in $o=\frac{i+2p-K}{s}+1$, we solve $32=\frac{64+2p-(7k-6)}{2}+1$. 
One solution is ${k=4, s=2, p=10 , d=7}$\\

8.2(a) Similarly we get ${k=4, s=4, p=0 , d=0}$\\

8.2(b) Again, using the above formula and ${k=8, s=4, p=0 , d=1}$, we calculate the output to $7\times 7$:  $o=\frac{i+2p-k}{s}+1=\frac{32+0-8}{4}+1=7$  \\

8.3(a) Solving $o=\frac{i+2p-k}{s}+1$, we get $4=\frac{8+0-k}{s}+1$, which can be solved by:\\ ${k=5, s=1, p=0 , d=1}$\\

8.3(b) With dilation $d=2$, the new kernel size will be $K=2k-1$. We solve $4=\frac{8+4-K}{s}+1=$ and pick one solution:\\
${k=5, s=1, p=2 , d=1}$\\

8.3(c) Similarly, a couple of solutions are:\\ 
${k=7, s=1, p=1 , d=1}$ or\\
${k=4, s=2, p=1 , d=1}$
}
