% (meta) 
% Exercise contributed by Louis-Guillaume Gagnon
% label: ch6

\Exercise{
\label{ex:heaviside}
Using the following definition of the derivative and the definition of the Heaviside step function:
$$
\frac{d}{dx}f(x) = 
\lim_{\epsilon\rightarrow 0}\ \frac{f(x + \epsilon) - f(x)}{\epsilon}
\qquad\quad
H(x)=\begin{cases}
1 & \text{if $x > 0$}\\
\frac{1}{2} & \text{if $x = 0$}\\
0 & \text{if $x < 0$}
\end{cases}
$$

\begin{enumerate}
\item Show that the derivative of the rectified linear unit $g(x) = \max\{0, x\}$, \textbf{wherever it exists}, is equal to the Heaviside step function.

\item Give two alternative definitions of $g(x)$ using $H(x)$.

\item Show that $H(x)$ can be well approximated by the sigmoid function $\sigma(x)=\frac{1}{1+e^{-k{x}}}$ asymptotically (i.e for large $k$), where $k$ is a parameter. 
%\item Show that the derivative of $H(x)$ is equal to $$\delta(x)=\begin{cases}
%\infty & \textnormal{if $x = 0$}\\
%0 & \textnormal{if $x \neq 0$.}
%\end{cases}$$

\staritem Although the Heaviside step function is not differentiable, we can define its {\bf distributional derivative}. 
For a function $F$, consider the functional $F[\phi] = \int_{\R}F(x)\phi(x)d x$, where $\phi$ is a smooth function (infinitely differentiable) with compact support ($\phi(x)= 0$ whenever  $|x| \geq A$, for some $A>0$). 

Show that whenever $F$ is differentiable, 
$F'[\phi] = - \int_{\R}F(x)\phi'(x)d x$.  
Using this formula as a definition in the case of non-differentiable functions, show that $H'[\phi]=\phi(0)$. 
($\delta[\phi]\doteq\phi(0)$ is known as the Dirac delta function.)
\end{enumerate}
}

\Answer{Q1.1:
$$
\lim_{\epsilon\rightarrow 0}\ \frac{g(x + \epsilon) - g(x)}{\epsilon}
=\lim_{\epsilon\rightarrow 0}\ \frac{max(0,x + \epsilon) - max(0,x)}{\epsilon}
=\lim_{\epsilon\rightarrow 0}\ \frac{x + \epsilon - x}{\epsilon}
=1 \;\;\;\text{for $x > 0$}
$$
As seen below, the right and left limits of the rectified linear unit are not equal, hence it is no differentiable at zero. This is due to the fact that both $x+\epsilon$ and $x$ are negative on the left side of zero but positive on the right side.
$$
\lim_{\epsilon\rightarrow 0^+}\ \frac{max(0,x + \epsilon) - max(0,x)}{\epsilon}=
\lim_{\epsilon\rightarrow 0^+}\ \frac{x + \epsilon - x}{\epsilon}
=1
$$
$$
\lim_{\epsilon\rightarrow 0^-}\ \frac{max(0,x + \epsilon) - max(0,x)}{\epsilon}=
\lim_{\epsilon\rightarrow 0
^-}\ \frac{0 - 0}{\epsilon}
=0& 
$$
Similarly, since $x+\epsilon$ and $x$ are negative on the left side of zero, we find that the derivative of the rectified linear unit is zero.
$$
\lim_{\epsilon\rightarrow 0}\ \frac{g(x + \epsilon) - g(x)}{\epsilon}
=\lim_{\epsilon\rightarrow 0}\ \frac{max(0,x + \epsilon) - max(0,x)}{\epsilon}
=\lim_{\epsilon\rightarrow 0}\ \frac{0-0}{\epsilon}
=0\;\;\text{for $x < 0$}
$$
In conclusion:\\
$$
\frac{d}{dx}g(x)=H(x)\;\text{ for $x \neq 0$}
$$
Q 1.2: Two alternative definitions of Relu could be: $g(x)=xH(x)$ and $g(x)=\int_0^x H(x) dx$.\\ 
Q 1.3: There are two asymptotic cases:\\
$$
\lim_{k\rightarrow +\infty}\sigma(x)=\lim_{k\rightarrow +\infty}\frac{1}{1+e^{-k{x}}}=\begin{cases}
\lim_{k\rightarrow +\infty}\frac{1}{1+e^{-\infty}}=1 & \text{if $x > 0$}\\
\lim_{k\rightarrow+\infty}\frac{1}{1+e^{\infty}}=0 & \text{if $x < 0$}
\end{cases}
$$
$$
\lim_{k\rightarrow -\infty}\sigma(x)=\lim_{k\rightarrow +\infty}\frac{1}{1+e^{-k{x}}}=\begin{cases}
\lim_{k\rightarrow -\infty}\frac{1}{1+e^{+\infty}}=0 & \text{if $x > 0$}\\
\lim_{k\rightarrow-\infty}\frac{1}{1+e^{-\infty}}=1 & \text{if $x < 0$}
\end{cases}
$$
In conclusion, when $k\rightarrow+\infty$, $\sigma(x)$ exactly mimics the behavior of the Heaviside function $H(x)$. \\
Q 1.4: Since the derivative of a function is another function, we define: $F'[x] =G(x)$. Using the definition of the functional/distributional derivative, we have:
$$G[\phi] =  \int_{\R}G(x)\phi(x)d x \Rightarrow F'[\phi]=   \int_{\R}F'(x)\phi(x)d x$$
Using integration by parts $\int udv=uv-\int vdu$ with $u=\phi(x)$ and $dv=F'(x)dx$, we have:\\
$$
F'[\phi]=   \int_{\R}F'(x)\phi(x)d x=F(x)\phi(x)|_{\R}-\int_{\R}F(x)\phi'(x)d x
=-\int_{\R}F(x)\phi'(x)d x
$$

$$
F(x)\phi(x)|_{\R}=F(x)\phi(x)|^{+\infty}_{-\infty}=F(x)\phi(x)|^{A}_{-A}=F(A)\phi(A)-F(-A)\phi(-A)=0
$$
Here we have used the fact that $\phi (x)$ has a compact support, as defined in the problem i.e that it $\phi(x)=0$ on the interval $|x|\geq A$ for some positive $A$.  
}