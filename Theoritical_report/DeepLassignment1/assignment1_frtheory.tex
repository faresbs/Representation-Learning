\documentclass[12pt]{article}
\usepackage[canadien]{babel} 
\usepackage[utf8]{inputenc}
\usepackage[T1]{fontenc}
\usepackage{fancyhdr}
\usepackage{graphicx}
\usepackage{enumerate}
\usepackage{amsmath}
\usepackage{amssymb}
\usepackage{subfigure}
\usepackage{amsmath}
\usepackage{amssymb}
\usepackage{bm}
\usepackage{url}
\usepackage{todonotes}

% If your paper is accepted, change the options for the package
% aistats2e as follows:
%
%\usepackage[accepted]{aistats2e}
%
% This option will print headings for the title of your paper and
% headings for the authors names, plus a copyright note at the end of
% the first column of the first page.
\setlength{\parindent}{0cm}
\addtolength{\oddsidemargin}{-2cm}
\addtolength{\evensidemargin}{-2cm}
\setlength{\textwidth}{17.78cm}
\addtolength{\topmargin}{-2.25cm}
\setlength{\textheight}{24.24cm}
\addtolength{\parskip}{5mm}
\pagestyle{fancy}

%************
%* COMMANDS *
%************

%%%%% NEW MATH DEFINITIONS %%%%%

% Mark sections of captions for referring to divisions of figures
\newcommand{\figleft}{{\em (Left)}}
\newcommand{\figcenter}{{\em (Center)}}
\newcommand{\figright}{{\em (Right)}}
\newcommand{\figtop}{{\em (Top)}}
\newcommand{\figbottom}{{\em (Bottom)}}
\newcommand{\captiona}{{\em (a)}}
\newcommand{\captionb}{{\em (b)}}
\newcommand{\captionc}{{\em (c)}}
\newcommand{\captiond}{{\em (d)}}

% Highlight a newly defined term
\newcommand{\newterm}[1]{{\bf #1}}


% Figure reference, lower-case.
\def\figref#1{figure~\ref{#1}}
% Figure reference, capital. For start of sentence
\def\Figref#1{Figure~\ref{#1}}
\def\twofigref#1#2{figures \ref{#1} and \ref{#2}}
\def\quadfigref#1#2#3#4{figures \ref{#1}, \ref{#2}, \ref{#3} and \ref{#4}}
% Section reference, lower-case.
\def\secref#1{section~\ref{#1}}
% Section reference, capital.
\def\Secref#1{Section~\ref{#1}}
% Reference to two sections.
\def\twosecrefs#1#2{sections \ref{#1} and \ref{#2}}
% Reference to three sections.
\def\secrefs#1#2#3{sections \ref{#1}, \ref{#2} and \ref{#3}}
% Reference to an equation, lower-case.
\def\eqref#1{equation~\ref{#1}}
% Reference to an equation, upper case
\def\Eqref#1{Equation~\ref{#1}}
% A raw reference to an equation---avoid using if possible
\def\plaineqref#1{\ref{#1}}
% Reference to a chapter, lower-case.
\def\chapref#1{chapter~\ref{#1}}
% Reference to an equation, upper case.
\def\Chapref#1{Chapter~\ref{#1}}
% Reference to a range of chapters
\def\rangechapref#1#2{chapters\ref{#1}--\ref{#2}}
% Reference to an algorithm, lower-case.
\def\algref#1{algorithm~\ref{#1}}
% Reference to an algorithm, upper case.
\def\Algref#1{Algorithm~\ref{#1}}
\def\twoalgref#1#2{algorithms \ref{#1} and \ref{#2}}
\def\Twoalgref#1#2{Algorithms \ref{#1} and \ref{#2}}
% Reference to a part, lower case
\def\partref#1{part~\ref{#1}}
% Reference to a part, upper case
\def\Partref#1{Part~\ref{#1}}
\def\twopartref#1#2{parts \ref{#1} and \ref{#2}}

\def\ceil#1{\lceil #1 \rceil}
\def\floor#1{\lfloor #1 \rfloor}
\def\1{\bm{1}}
\newcommand{\train}{\mathcal{D}}
\newcommand{\valid}{\mathcal{D_{\mathrm{valid}}}}
\newcommand{\test}{\mathcal{D_{\mathrm{test}}}}

\def\eps{{\epsilon}}


% Random variables
\def\reta{{\textnormal{$\eta$}}}
\def\ra{{\textnormal{a}}}
\def\rb{{\textnormal{b}}}
\def\rc{{\textnormal{c}}}
\def\rd{{\textnormal{d}}}
\def\re{{\textnormal{e}}}
\def\rf{{\textnormal{f}}}
\def\rg{{\textnormal{g}}}
\def\rh{{\textnormal{h}}}
\def\ri{{\textnormal{i}}}
\def\rj{{\textnormal{j}}}
\def\rk{{\textnormal{k}}}
\def\rl{{\textnormal{l}}}
% rm is already a command, just don't name any random variables m
\def\rn{{\textnormal{n}}}
\def\ro{{\textnormal{o}}}
\def\rp{{\textnormal{p}}}
\def\rq{{\textnormal{q}}}
\def\rr{{\textnormal{r}}}
\def\rs{{\textnormal{s}}}
\def\rt{{\textnormal{t}}}
\def\ru{{\textnormal{u}}}
\def\rv{{\textnormal{v}}}
\def\rw{{\textnormal{w}}}
\def\rx{{\textnormal{x}}}
\def\ry{{\textnormal{y}}}
\def\rz{{\textnormal{z}}}

% Random vectors
\def\rvepsilon{{\mathbf{\epsilon}}}
\def\rvtheta{{\mathbf{\theta}}}
\def\rva{{\mathbf{a}}}
\def\rvb{{\mathbf{b}}}
\def\rvc{{\mathbf{c}}}
\def\rvd{{\mathbf{d}}}
\def\rve{{\mathbf{e}}}
\def\rvf{{\mathbf{f}}}
\def\rvg{{\mathbf{g}}}
\def\rvh{{\mathbf{h}}}
\def\rvu{{\mathbf{i}}}
\def\rvj{{\mathbf{j}}}
\def\rvk{{\mathbf{k}}}
\def\rvl{{\mathbf{l}}}
\def\rvm{{\mathbf{m}}}
\def\rvn{{\mathbf{n}}}
\def\rvo{{\mathbf{o}}}
\def\rvp{{\mathbf{p}}}
\def\rvq{{\mathbf{q}}}
\def\rvr{{\mathbf{r}}}
\def\rvs{{\mathbf{s}}}
\def\rvt{{\mathbf{t}}}
\def\rvu{{\mathbf{u}}}
\def\rvv{{\mathbf{v}}}
\def\rvw{{\mathbf{w}}}
\def\rvx{{\mathbf{x}}}
\def\rvy{{\mathbf{y}}}
\def\rvz{{\mathbf{z}}}

% Elements of random vectors
\def\erva{{\textnormal{a}}}
\def\ervb{{\textnormal{b}}}
\def\ervc{{\textnormal{c}}}
\def\ervd{{\textnormal{d}}}
\def\erve{{\textnormal{e}}}
\def\ervf{{\textnormal{f}}}
\def\ervg{{\textnormal{g}}}
\def\ervh{{\textnormal{h}}}
\def\ervi{{\textnormal{i}}}
\def\ervj{{\textnormal{j}}}
\def\ervk{{\textnormal{k}}}
\def\ervl{{\textnormal{l}}}
\def\ervm{{\textnormal{m}}}
\def\ervn{{\textnormal{n}}}
\def\ervo{{\textnormal{o}}}
\def\ervp{{\textnormal{p}}}
\def\ervq{{\textnormal{q}}}
\def\ervr{{\textnormal{r}}}
\def\ervs{{\textnormal{s}}}
\def\ervt{{\textnormal{t}}}
\def\ervu{{\textnormal{u}}}
\def\ervv{{\textnormal{v}}}
\def\ervw{{\textnormal{w}}}
\def\ervx{{\textnormal{x}}}
\def\ervy{{\textnormal{y}}}
\def\ervz{{\textnormal{z}}}

% Random matrices
\def\rmA{{\mathbf{A}}}
\def\rmB{{\mathbf{B}}}
\def\rmC{{\mathbf{C}}}
\def\rmD{{\mathbf{D}}}
\def\rmE{{\mathbf{E}}}
\def\rmF{{\mathbf{F}}}
\def\rmG{{\mathbf{G}}}
\def\rmH{{\mathbf{H}}}
\def\rmI{{\mathbf{I}}}
\def\rmJ{{\mathbf{J}}}
\def\rmK{{\mathbf{K}}}
\def\rmL{{\mathbf{L}}}
\def\rmM{{\mathbf{M}}}
\def\rmN{{\mathbf{N}}}
\def\rmO{{\mathbf{O}}}
\def\rmP{{\mathbf{P}}}
\def\rmQ{{\mathbf{Q}}}
\def\rmR{{\mathbf{R}}}
\def\rmS{{\mathbf{S}}}
\def\rmT{{\mathbf{T}}}
\def\rmU{{\mathbf{U}}}
\def\rmV{{\mathbf{V}}}
\def\rmW{{\mathbf{W}}}
\def\rmX{{\mathbf{X}}}
\def\rmY{{\mathbf{Y}}}
\def\rmZ{{\mathbf{Z}}}

% Elements of random matrices
\def\ermA{{\textnormal{A}}}
\def\ermB{{\textnormal{B}}}
\def\ermC{{\textnormal{C}}}
\def\ermD{{\textnormal{D}}}
\def\ermE{{\textnormal{E}}}
\def\ermF{{\textnormal{F}}}
\def\ermG{{\textnormal{G}}}
\def\ermH{{\textnormal{H}}}
\def\ermI{{\textnormal{I}}}
\def\ermJ{{\textnormal{J}}}
\def\ermK{{\textnormal{K}}}
\def\ermL{{\textnormal{L}}}
\def\ermM{{\textnormal{M}}}
\def\ermN{{\textnormal{N}}}
\def\ermO{{\textnormal{O}}}
\def\ermP{{\textnormal{P}}}
\def\ermQ{{\textnormal{Q}}}
\def\ermR{{\textnormal{R}}}
\def\ermS{{\textnormal{S}}}
\def\ermT{{\textnormal{T}}}
\def\ermU{{\textnormal{U}}}
\def\ermV{{\textnormal{V}}}
\def\ermW{{\textnormal{W}}}
\def\ermX{{\textnormal{X}}}
\def\ermY{{\textnormal{Y}}}
\def\ermZ{{\textnormal{Z}}}

% Vectors
\def\vzero{{\bm{0}}}
\def\vone{{\bm{1}}}
\def\vmu{{\bm{\mu}}}
\def\vtheta{{\bm{\theta}}}
\def\va{{\bm{a}}}
\def\vb{{\bm{b}}}
\def\vc{{\bm{c}}}
\def\vd{{\bm{d}}}
\def\ve{{\bm{e}}}
\def\vf{{\bm{f}}}
\def\vg{{\bm{g}}}
\def\vh{{\bm{h}}}
\def\vi{{\bm{i}}}
\def\vj{{\bm{j}}}
\def\vk{{\bm{k}}}
\def\vl{{\bm{l}}}
\def\vm{{\bm{m}}}
\def\vn{{\bm{n}}}
\def\vo{{\bm{o}}}
\def\vp{{\bm{p}}}
\def\vq{{\bm{q}}}
\def\vr{{\bm{r}}}
\def\vs{{\bm{s}}}
\def\vt{{\bm{t}}}
\def\vu{{\bm{u}}}
\def\vv{{\bm{v}}}
\def\vw{{\bm{w}}}
\def\vx{{\bm{x}}}
\def\vy{{\bm{y}}}
\def\vz{{\bm{z}}}

% Elements of vectors
\def\evalpha{{\alpha}}
\def\evbeta{{\beta}}
\def\evepsilon{{\epsilon}}
\def\evlambda{{\lambda}}
\def\evomega{{\omega}}
\def\evmu{{\mu}}
\def\evpsi{{\psi}}
\def\evsigma{{\sigma}}
\def\evtheta{{\theta}}
\def\eva{{a}}
\def\evb{{b}}
\def\evc{{c}}
\def\evd{{d}}
\def\eve{{e}}
\def\evf{{f}}
\def\evg{{g}}
\def\evh{{h}}
\def\evi{{i}}
\def\evj{{j}}
\def\evk{{k}}
\def\evl{{l}}
\def\evm{{m}}
\def\evn{{n}}
\def\evo{{o}}
\def\evp{{p}}
\def\evq{{q}}
\def\evr{{r}}
\def\evs{{s}}
\def\evt{{t}}
\def\evu{{u}}
\def\evv{{v}}
\def\evw{{w}}
\def\evx{{x}}
\def\evy{{y}}
\def\evz{{z}}

% Matrix
\def\mA{{\bm{A}}}
\def\mB{{\bm{B}}}
\def\mC{{\bm{C}}}
\def\mD{{\bm{D}}}
\def\mE{{\bm{E}}}
\def\mF{{\bm{F}}}
\def\mG{{\bm{G}}}
\def\mH{{\bm{H}}}
\def\mI{{\bm{I}}}
\def\mJ{{\bm{J}}}
\def\mK{{\bm{K}}}
\def\mL{{\bm{L}}}
\def\mM{{\bm{M}}}
\def\mN{{\bm{N}}}
\def\mO{{\bm{O}}}
\def\mP{{\bm{P}}}
\def\mQ{{\bm{Q}}}
\def\mR{{\bm{R}}}
\def\mS{{\bm{S}}}
\def\mT{{\bm{T}}}
\def\mU{{\bm{U}}}
\def\mV{{\bm{V}}}
\def\mW{{\bm{W}}}
\def\mX{{\bm{X}}}
\def\mY{{\bm{Y}}}
\def\mZ{{\bm{Z}}}
\def\mBeta{{\bm{\beta}}}
\def\mPhi{{\bm{\Phi}}}
\def\mLambda{{\bm{\Lambda}}}
\def\mSigma{{\bm{\Sigma}}}

% Tensor
\DeclareMathAlphabet{\mathsfit}{\encodingdefault}{\sfdefault}{m}{sl}
\SetMathAlphabet{\mathsfit}{bold}{\encodingdefault}{\sfdefault}{bx}{n}
\newcommand{\tens}[1]{\bm{\mathsfit{#1}}}
\def\tA{{\tens{A}}}
\def\tB{{\tens{B}}}
\def\tC{{\tens{C}}}
\def\tD{{\tens{D}}}
\def\tE{{\tens{E}}}
\def\tF{{\tens{F}}}
\def\tG{{\tens{G}}}
\def\tH{{\tens{H}}}
\def\tI{{\tens{I}}}
\def\tJ{{\tens{J}}}
\def\tK{{\tens{K}}}
\def\tL{{\tens{L}}}
\def\tM{{\tens{M}}}
\def\tN{{\tens{N}}}
\def\tO{{\tens{O}}}
\def\tP{{\tens{P}}}
\def\tQ{{\tens{Q}}}
\def\tR{{\tens{R}}}
\def\tS{{\tens{S}}}
\def\tT{{\tens{T}}}
\def\tU{{\tens{U}}}
\def\tV{{\tens{V}}}
\def\tW{{\tens{W}}}
\def\tX{{\tens{X}}}
\def\tY{{\tens{Y}}}
\def\tZ{{\tens{Z}}}


% Graph
\def\gA{{\mathcal{A}}}
\def\gB{{\mathcal{B}}}
\def\gC{{\mathcal{C}}}
\def\gD{{\mathcal{D}}}
\def\gE{{\mathcal{E}}}
\def\gF{{\mathcal{F}}}
\def\gG{{\mathcal{G}}}
\def\gH{{\mathcal{H}}}
\def\gI{{\mathcal{I}}}
\def\gJ{{\mathcal{J}}}
\def\gK{{\mathcal{K}}}
\def\gL{{\mathcal{L}}}
\def\gM{{\mathcal{M}}}
\def\gN{{\mathcal{N}}}
\def\gO{{\mathcal{O}}}
\def\gP{{\mathcal{P}}}
\def\gQ{{\mathcal{Q}}}
\def\gR{{\mathcal{R}}}
\def\gS{{\mathcal{S}}}
\def\gT{{\mathcal{T}}}
\def\gU{{\mathcal{U}}}
\def\gV{{\mathcal{V}}}
\def\gW{{\mathcal{W}}}
\def\gX{{\mathcal{X}}}
\def\gY{{\mathcal{Y}}}
\def\gZ{{\mathcal{Z}}}

% Sets
\def\sA{{\mathbb{A}}}
\def\sB{{\mathbb{B}}}
\def\sC{{\mathbb{C}}}
\def\sD{{\mathbb{D}}}
% Don't use a set called E, because this would be the same as our symbol
% for expectation.
\def\sF{{\mathbb{F}}}
\def\sG{{\mathbb{G}}}
\def\sH{{\mathbb{H}}}
\def\sI{{\mathbb{I}}}
\def\sJ{{\mathbb{J}}}
\def\sK{{\mathbb{K}}}
\def\sL{{\mathbb{L}}}
\def\sM{{\mathbb{M}}}
\def\sN{{\mathbb{N}}}
\def\sO{{\mathbb{O}}}
\def\sP{{\mathbb{P}}}
\def\sQ{{\mathbb{Q}}}
\def\sR{{\mathbb{R}}}
\def\sS{{\mathbb{S}}}
\def\sT{{\mathbb{T}}}
\def\sU{{\mathbb{U}}}
\def\sV{{\mathbb{V}}}
\def\sW{{\mathbb{W}}}
\def\sX{{\mathbb{X}}}
\def\sY{{\mathbb{Y}}}
\def\sZ{{\mathbb{Z}}}

% Entries of a matrix
\def\emLambda{{\Lambda}}
\def\emA{{A}}
\def\emB{{B}}
\def\emC{{C}}
\def\emD{{D}}
\def\emE{{E}}
\def\emF{{F}}
\def\emG{{G}}
\def\emH{{H}}
\def\emI{{I}}
\def\emJ{{J}}
\def\emK{{K}}
\def\emL{{L}}
\def\emM{{M}}
\def\emN{{N}}
\def\emO{{O}}
\def\emP{{P}}
\def\emQ{{Q}}
\def\emR{{R}}
\def\emS{{S}}
\def\emT{{T}}
\def\emU{{U}}
\def\emV{{V}}
\def\emW{{W}}
\def\emX{{X}}
\def\emY{{Y}}
\def\emZ{{Z}}
\def\emSigma{{\Sigma}}

% entries of a tensor
% Same font as tensor, without \bm wrapper
\newcommand{\etens}[1]{\mathsfit{#1}}
\def\etLambda{{\etens{\Lambda}}}
\def\etA{{\etens{A}}}
\def\etB{{\etens{B}}}
\def\etC{{\etens{C}}}
\def\etD{{\etens{D}}}
\def\etE{{\etens{E}}}
\def\etF{{\etens{F}}}
\def\etG{{\etens{G}}}
\def\etH{{\etens{H}}}
\def\etI{{\etens{I}}}
\def\etJ{{\etens{J}}}
\def\etK{{\etens{K}}}
\def\etL{{\etens{L}}}
\def\etM{{\etens{M}}}
\def\etN{{\etens{N}}}
\def\etO{{\etens{O}}}
\def\etP{{\etens{P}}}
\def\etQ{{\etens{Q}}}
\def\etR{{\etens{R}}}
\def\etS{{\etens{S}}}
\def\etT{{\etens{T}}}
\def\etU{{\etens{U}}}
\def\etV{{\etens{V}}}
\def\etW{{\etens{W}}}
\def\etX{{\etens{X}}}
\def\etY{{\etens{Y}}}
\def\etZ{{\etens{Z}}}

% The true underlying data generating distribution
\newcommand{\pdata}{p_{\rm{data}}}
% The empirical distribution defined by the training set
\newcommand{\ptrain}{\hat{p}_{\rm{data}}}
\newcommand{\Ptrain}{\hat{P}_{\rm{data}}}
% The model distribution
\newcommand{\pmodel}{p_{\rm{model}}}
\newcommand{\Pmodel}{P_{\rm{model}}}
\newcommand{\ptildemodel}{\tilde{p}_{\rm{model}}}
% Stochastic autoencoder distributions
\newcommand{\pencode}{p_{\rm{encoder}}}
\newcommand{\pdecode}{p_{\rm{decoder}}}
\newcommand{\precons}{p_{\rm{reconstruct}}}

\newcommand{\laplace}{\mathrm{Laplace}} % Laplace distribution

\newcommand{\E}{\mathbb{E}}
\newcommand{\Ls}{\mathcal{L}}
\newcommand{\R}{\mathbb{R}}
\newcommand{\emp}{\tilde{p}}
\newcommand{\lr}{\alpha}
\newcommand{\reg}{\lambda}
\newcommand{\rect}{\mathrm{rectifier}}
\newcommand{\softmax}{\mathrm{softmax}}
\newcommand{\sigmoid}{\sigma}
\newcommand{\softplus}{\zeta}
\newcommand{\KL}{D_{\mathrm{KL}}}
\newcommand{\Var}{\mathrm{Var}}
\newcommand{\standarderror}{\mathrm{SE}}
\newcommand{\Cov}{\mathrm{Cov}}
% Wolfram Mathworld says $L^2$ is for function spaces and $\ell^2$ is for vectors
% But then they seem to use $L^2$ for vectors throughout the site, and so does
% wikipedia.
\newcommand{\normlzero}{L^0}
\newcommand{\normlone}{L^1}
\newcommand{\normltwo}{L^2}
\newcommand{\normlp}{L^p}
\newcommand{\normmax}{L^\infty}

\newcommand{\parents}{Pa} % See usage in notation.tex. Chosen to match Daphne's book.

\DeclareMathOperator*{\argmax}{arg\,max}
\DeclareMathOperator*{\argmin}{arg\,min}

\DeclareMathOperator{\sign}{sign}
\DeclareMathOperator{\Tr}{Tr}
\let\ab\allowbreak


%%% new
\newcommand{\diag}{\mathop{\mathrm{diag}}\nolimits}


\newif\ifexercise
\exercisetrue
%\exercisefalse

\newif\ifsolution
\solutiontrue
%\solutionfalse

\usepackage{booktabs}
\usepackage[chapter]{algorithm}
\usepackage{algorithmic}
% Include chapter number in algorithm number
\renewcommand{\thealgorithm}{\arabic{chapter}.\arabic{algorithm}}

\usepackage{amsthm}
\theoremstyle{definition}
\newtheorem{exercise}{Question}%[chapter]
\newtheorem{answer}{Answer} % asterisk to remove ordering
\newcommand{\Exercise}[1]{
\begin{exercise}
\ifexercise#1\fi
\end{exercise}
}
\newcommand{\Answer}[1]{
\ifsolution\begin{answer}#1\end{answer}\fi
}

\newif\ifexercise
\exercisetrue
%\exercisefalse

\newif\ifsolution
\solutiontrue
%\solutionfalse

\usepackage{enumitem}
\newcommand{\staritem}{
\addtocounter{enumi}{1}
\item[$\phantom{x}^{*}$\theenumi]}
\setlist[enumerate,1]{leftmargin=*, label=\arabic*.}

\begin{document}


\fancyhead{}
\fancyfoot{}

\fancyhead[L]{
  \begin{tabular}[b]{l}
    IFT6135-H2019  \\
    Prof: Aaron Courville \\
  \end{tabular}
}
\fancyhead[R]{
  \begin{tabular}[b]{r}
    Devoir 1, Partie théorique  \\
    Multilayer Perceptrons and Convolutional Neural Networks \\
  \end{tabular}
}

\vspace{1cm}

{\bf A rendre le: 16 février 2019}\\

\vspace{-0.5cm}
\underline{Instructions}
%Laissez des traces de votre démarche pour toutes les questions! \\
\renewcommand{\labelitemi}{\textbullet}
\begin{itemize}
\item \emph{Montrez votre démarche pour toutes les questions!}
\item \emph{Utilisez un logiciel de traitement de texte comme LaTeX.}
\item \emph{Vous devez soumettre toutes vos réponses sur la page Gradescope du cours.}
\end{itemize}

% (meta) 
% Exercise contributed by Louis-Guillaume Gagnon
% Translation by Salem Lahlou
% label: ch6

\Exercise{
\label{ex:heaviside}
En utilisant les définitions de la dérivée et de la fonction de \textit{Heaviside} (fonction marche d'escalier) suivantes:
$$
\frac{d}{dx}f(x) = 
\lim_{\epsilon\rightarrow 0}\ \frac{f(x + \epsilon) - f(x)}{\epsilon}
\qquad\quad
H(x)=\begin{cases}
1 & \text{if $x > 0$}\\
\frac{1}{2} & \text{if $x = 0$}\\
0 & \text{if $x < 0$}
\end{cases}
$$

\begin{enumerate}
\item Montrez que la dérivée de la fonction d'activation ReLU (Unité de Rectification Linéaire) $g(x) = \max\{0, x\}$, \textbf{partout où elle existe} est égale a la fonction de Heaviside.


\item Donnez deux définitions alternatives de $g(x)$ en utilisant $H(x)$.

\item Montrez qu'on peut bien approximer $H(x)$ en utilisant la fonction logistique (la sigmoïde) $\sigma(x)=\frac{1}{1+e^{-k{x}}}$ asymptotiquement (c.-à-d. pour des valeurs de $k$ large), $k$ étant un paramètre. 
%\item Show that the derivative of $H(x)$ is equal to $$\delta(x)=\begin{cases}
%\infty & \textnormal{if $x = 0$}\\
%0 & \textnormal{if $x \neq 0$.}
%\end{cases}$$

\staritem Même si la fonction de Heaviside est non-dérivable, on peut définir sa {\bf dérivée distributionelle}. 
Pour une fonction $F$, considérez la fonctionnelle $F[\phi] = \int_{\R}F(x)\phi(x)d x$,  $\phi$ étant une fonction lisse (indéfiniment dérivable, c.-à-d. dans $\mathcal{C}^\infty$) à support compact ($\phi(x)= 0$ quand  $|x| \geq A$, pour un certain $A>0$). 

Montrez que si $F$ est dérivable, alors
$F'[\phi] = - \int_{\R}F(x)\phi'(x)d x$. En utilisant cette formule comme une définition de la dérivée distributionelle dans le cas des fonctions non-dérivables, montrez que
 $H'[\phi]=\phi(0)$. 
($\delta[\phi]\doteq\phi(0)$ est la fonction delta de Dirac (ou distribution de Dirac))
\end{enumerate}
}

\Answer{
${}$%placeholder
}
% (meta) 
% Exercise contributed by Louis-Guillaume Gagnon
% Translation by Salem Lahlou
% label: ch6

\Exercise{
\label{ex:grad_softmax}
Soit $x \in \R^n$ un vecteur.
On rappelle les définitions de la fonction softmax (fonction exponentielle normalisée): $S: \vx \in \R^n \mapsto S(\vx) \in \R^n$ tel que $S(\vx)_i=\frac{e^{\vx_i}}{\sum_j e^{\vx_j}}$ et de la fonction diagonale: $\text{diag}(\vx)_{ij}=\vx_i$ si $i=j$ et $\text{diag}(\vx)_{ij}=0$ si $i\neq j$;
et du symbole de Kronecker: $\delta_{ij}=1$ if $i=j$ and $\delta_{ij}=0$ if $i\neq j$.  

\begin{enumerate}
\item Montrez que la dérivée de la fonction softmax est $\frac{d S(\vx)_i}{d \vx_j}=S(\vx)_i\left(\delta_{ij}-S(\vx)_j\right)$.

\item Exprimez la matrice Jacobienne $\frac{\partial S(\vx)}{\partial \vx}$ en utilisant la notation matricielle/vectorielle. Utilisez $\text{diag}(\cdot)$.

\item Calculez la matrice Jacobienne de la fonction logistique $\sigma(\vx) = 1/(1 + e^{-\vx})$.

\item Soient $\vy, \vx \in \R^n$ des vecteurs, tels que $\vy = f(\vx)$. Soit $L$ une fonction de coût, differentiable.
D'après le théorème de dérivation des fonctions composées (règle de dérivation en chaîne), $ \nabla_\vx L = (\frac{\partial \vy}{\partial \vx})^\top  \nabla_\vy L$, dont le calcul a une complexité en temps de $\gO(n^2)$, en général.
Montrer que si $f(\vx)=\sigma(\vx)$ ou $f(\vx)=S(\vx)$, alors la multiplication matrice-vecteur précédente peut être simplifiée, et évaluée en utilisant $\gO(n)$ opérations.
\end{enumerate}
}

\Answer{
${}$%placeholder
}
% (meta) 
% Exercise contributed by Antoine Lefebvre-Brossard
% Translation by Salem Lahlou
% label: ch6

\Exercise{
\label{ex:properties_softmax}
On rappelle la définition de la fonction softmax: $S(\vx)_i={e^{\vx_i}}/{\sum_j e^{\vx_j}}$.
\begin{enumerate}
\item Montrez que la fonction softmax est invariante aux translations, c'est-à-dire : $S(\vx+c) = S(\vx)$, où $c$ est une constante.

\item Montrez que la fonction softmax n'est pas invariante aux multiplications scalaires.
On définit $S_c(\vx)=S(c\vx)$ où $c\geq0$. Quels seraient les effets si on choisissait $c = 0$ ou $c \rightarrow \infty$. 

\item Soit $\vx \in \R^2$ un vecteur. On peut représenter une probabilité catégorique sur deux classes en utilisant la fonction softmax. Montrez que $S(x)$ peut être reparamétrisée en utilisant la fonction sigmoïde, c'est-à-dire : $S(\vx)=[\sigma(z), 1-\sigma(z)]^\top$ où $z$ est un scalaire, qu'il faut exprimer en fonction de $\vx$.

\item Soit $\vx \in \R^K$ un vecteur ($K\geq2$).
Montrez que $S(\vx)$ peut être représentée avec $K-1$ paramètres, c.-à-d.
$S(\vx)=S([0, y_1, y_2, ..., y_{K-1}]^\top)$ où $y_{i}$ sont des scalaires, à exprimer en fonction de $\vx$ pour $i\in\{1,...,K-1\}$. 
\end{enumerate}
}

\Answer{
${}$%placeholder
}
% (meta)
% Exercise contributed by Julian Zaidi
% Translation by Salem Lahlou
% label: ch6

\Exercise{
\label{ex:activation_transform}
On considère un réseau de neurones à deux couches $y:\R^D\rightarrow\R^K$ de la forme suivante:
$$
y(x, \Theta, \sigma)_k=\sum_{j=1}^M \omega_{kj}^{(2)}\sigma\left(\sum_{i=1}^D \omega_{ji}^{(1)}x_i+\omega_{j0}^{(1)}\right)+\omega_{k0}^{(2)}
$$
pour $1\leq k\leq K$. Les paramètres du réseau sont $\Theta=(\omega^{(1)},\omega^{(2)})$. La fonction d'activation utilisée est $\sigma$, la fonction logistique. 
Montrez qu'il existe un réseau équivalent de la même forme, avec des paramètres  $\Theta'=(\tilde{\omega}^{(1)},\tilde{\omega}^{(2)})$, avec la fonction d'activation $\tanh$, tel que $y(x, \Theta', \tanh)=y(x, \Theta, \sigma)$ pour tout $x \in \R^D$. Exprimez $\Theta'$ en fonction de $\Theta$.
}

\Answer{
${}$%placeholder
}

% (meta)
% Exercise contributed by Aristide Baratin
% label: ch6

\Exercise{
\label{ex:finite_sample_uat}
Given $N \in \sZ^+$, we want to show that for any $f:\R^n \to \R^m$ and any sample set $\gS\subset \R^n$ of size $N$, there is a set of parameters for a two-layer network such that the output $y(\vx)$ matches $f(\vx)$ for all $\vx \in \gS$.
That is, we want to interpolate $f$ with $y$ on any finite set of samples $\gS$. 
\begin{enumerate}
\item Write the generic form of the function $y: \R^n \to \R^m$ defined by a 2-layer network with $N-1$ hidden units, with linear output and activation function $\phi$, in terms of its weights and biases $(\mW^{(1)}, \vb^{(1)})$ and $(\mW^{(2)}, \vb^{(2)})$.
\item In what follows, we will restrict $\mW^{(1)}$ to be $\mW^{(1)} = [\vw, \cdots, \vw]^T$ for some $\vw \in \R^n$ (so the rows of  $\mW^{(1)}$ are all the same).
Show that the interpolation problem on the sample set $\gS=\{\vx^{(1)}, \cdots \vx^{(N)}\} \subset \R^n$ can be reduced to solving a matrix equation: 
$\mM\tilde{\mW}^{(2)}=\mF$,
where $\tilde{\mW}^{(2)}$ and $\mF$ are both $N\times m$, given by
$$\tilde{\mW}^{(2)}=[\mW^{(2)}, \vb^{(2)}]^\top \qquad\qquad \mF=[f(\vx^{(1)}), \cdots, f(\vx^{(N)})]^\top$$
Express the $N \times N$ matrix $\mM$ in terms of $\vw$, $\vb^{(1)}$, $\phi$ and $\vx^{(i)}$.
\staritem {\bf Proof with Relu activation.} 
Assume $\vx^{(i)}$ are all distinct. 
Choose $\vw$ such that $\vw^\top \vx^{(i)}$ are also all distinct (Try to prove the existence of such a $\vw$, although this is not required for the assignment - See Assignment 0). 
Set $\vb^{(1)}_j = -\vw^\top \vx^{(j)} + \epsilon$, where $\epsilon >0$. 
Find a value of $\epsilon$ such that $\mM$ is triangular with non-zero diagonal elements.
Conclude.
(Hint: assume an ordering of $\vw^\top\vx^{(i)}$.)
\staritem {\bf Proof with sigmoid-like activations}. 
Assume $\phi$ is continuous, bounded, $\phi(-\infty)=0$ and $\phi(0)>0$.
Decompose $\vw$ as $\vw=\lambda\vu$. 
Set $\vb^{(1)}_j = -\lambda \vu^\top \vx^{(j)}$.
Fixing $\vu$, show that $\lim_{\lambda\to +\infty} {\mM}$ is triangular with non-zero diagonal elements. Conclude. 
(Note that doing so preserves the distinctness of $\vw^\top \vx^{(i)}$.)
\end{enumerate}
}

\Answer{
${}$%placeholder
\begin{enumerate}
\item Let's $\vx$ be is the input vector of the NN, which is of dimension $n\times 1$. 
The dimensions of the weights are:
\begin{enumerate}
\item[-] $\mW^{(1)}$ is $N-1\times n$
\item[-] $\vb^{(1)}$ is $N-1\times 1$
\item[-] $\mW^{(2)}$ is $m\times N-1$
\item[-] $\vb^{(2)}$ is $m\times 1$
\end{enumerate}
The activation of the first layer is defined by:
$h^{(1)} = \phi(\mW^{(1)} \vx + \vb^{(1)}))$
\newline
As the output activation is a linear function of the hidden layer, the output function is then:
$$y(\vx) = \mW^{(2)} \phi(\mW^{(1)} \vx + \vb^{(1)})) + \vb^{(2)})$$
\item From the formulation above, we can generalize to $N$ samples as follows:
$$y(\gS) = \mW^{(2)} \phi(\mW^{(1)} \mX + \vb^{(1)})) + \vb^{(2)})$$
where $\mX = [\vx^{(1)}, \cdots, \vx^{(N)}]$, is the matrix of input with dimension $n\times N$.

Specifically we have: 
$$\mW^{(1)} \vX = [\vw, \cdots, \vw]^T \vX$$
$$    = {\bm{1}}_{N-1} \vw^T \vX$$
where ${\bm{1}}_{N-1}$ is a vector of ones with dimension of $N-1 \times 1$. The result is a $N-1 \times N$ matrix. The addition of $\vb^{(1)}$ and the activation function $\phi$ will not change this dimension.
So the interpolation problem on the sample $\gS$ is solved by the equation:

\begin{equation}
\begin{split}
y(\gS) & = f(\gS) \\
& = [f(\vx^{(1)}), \cdots, f(\vx^{(N)})]^\top \\
& = \mF
\end{split}
\end{equation}

using the formulation of the NN above, we got:

$$\mW^{(2)} \phi(\bm{1}_{N-1} \vw^T \mX + \vb^{(1)})) + \vb^{(2)})=\mF$$

given the definition of $\tilde{\mW}^{(2)}$, we got

$$
\begin{vmatrix}
\phi(\bm{1}_{N-1} \vw^T \vx^{(1)}+\vb^{(1)})^T&1\\
\cdots\\
\phi(\bm{1}_{N-1} \vw^T \vx^{(N)}+\vb^{(1)})^T&1\\
\end{vmatrix}
\times
\begin{vmatrix}
{\mW^{(2)}}^\top\\
{\vb^{(2)}}^\top 
\end{vmatrix}
= \mF$$
or
$$\mM [\mW^{(2)}, \vb^{(2)}]^\top = \mF$$

where:
$$
\mM =
\begin{vmatrix}
\phi(\bm{1}_{N-1} \vw^T \vx^{(1)}+\vb^{(1)})^T&1\\
\cdots\\
\phi(\bm{1}_{N-1} \vw^T \vx^{(N)}+\vb^{(1)})^T&1\\
\end{vmatrix}$$

\item  Let's replace $\vb^{(1)}_j = -\vw^\top \vx^{(j)} + \epsilon$ in the matrix $\mM$ defined earlier:
$$\mM = 
\begin{vmatrix}
\phi(\bm{1}_{N-1} \vw^T \vx^{(1)}-\vw^\top \vx^{(1)} + \epsilon)^T&1\\
\cdots\\
\phi(\bm{1}_{N-1} \vw^T \vx^{(N)}-\vw^\top \vx^{(N)} + \epsilon)^T&1\\
\end{vmatrix}
$$

$$= 
\begin{vmatrix}
\phi(\bm{1}_{N-1} \epsilon)^T&1\\
\cdots\\
\phi(\bm{1}_{N-1} \epsilon)^T&1\\
\end{vmatrix}
$$

So, in order to make $\mM$ triangular, assuming that $\phi$ is a Relu function:

$\phi({\bm{1}}_{N-1} \epsilon) = 0$ whenever i<j ??????

\item Let's replace $\vw$ by $\lambda\vu$ and $\vb^{(1)}_j$ by $-\lambda \vu^\top \vx^{(j)}$ in $\mM$:
$$\mM = 
\begin{vmatrix}
\phi(\bm{1}_{N-1} \lambda\vu^T \vx^{(1)}-\lambda \vu^\top \vx^{(1)})^T&1\\
\cdots\\
\phi(\bm{1}_{N-1} \lambda\vu^T \vx^{(N)}-\lambda \vu^\top \vx^{(N)})^T&1\\
\end{vmatrix}
$$

???

\end{enumerate}

{ANSWER Nr 2, Parviz}\\
Q.1
$$\mY = \mW^{(2)} [\phi(\mW^{(1)} \mX + \vb^{(1)})] + \vb^{(2)}$$
Q.2: Define the data matrix (pseudo-data matrix) and a pseudo-weight matrix as
$$
\tilde{\mat{X}}=
  \begin{bmatrix}
  \vx^{(1)}_1&\vx^{(2)}_1&\dots&\vx^{(N)}_1\\
  \vdots&\vdots&\dots&\vdots\\
  \vx^{(1)}_n&\vx^{(2)}_n&\dots&\vx^{(N)}_n\\
  1&1&\dots&1  
  \end{bmatrix},\;\;\;
  \tilde{\mat{W}}^{(1)}=
  \begin{bmatrix}
  \vw^{(1)}_{11}&\vw^{(1)}_{12}&\dots&\vw^{(1)}_{1n}&\vb^{(1)}_1\\
  \vw^{(1)}_{21}&\vw^{(1)}_{22}&\dots&\vw^{(1)}_{2n}&\vb^{(1)}_2\\
  \vdots&\vdots&\dots&\vdots&\vdots\\
 
  \vw^{(1)}_{N-1,1}&\vw^{(1)}_{N-1,2}&\dots&\vw^{(1)}_{N-1,n}&\vb^{(1)}_{N-1}
  \end{bmatrix}
$$
We will have:
$$
\phi(\mW^{(1)} \mX + \vb^{(1)})=
$$
}

% (meta)
% Exercise contributed by Philip Paquette
% label: ch9

\Exercise{
\label{ex:1dconv_eval}
Compute the \textit{full}, \textit{valid}, and \textit{same} convolution (with kernel flipping) for the following 1D matrices: $\begin{bmatrix}1,2,3,4\end{bmatrix} * \begin{bmatrix}1,0,2\end{bmatrix}$
}

\Answer{
${}$%placeholder
Write your answer here.
}
% (meta)
% Exercise contributed by Mariane Maynard
% label: ch9

\Exercise{
\label{ex:convolution_shape}
Consider a convolutional neural network. 
Assume the input is a colorful image of size $256 \times 256$ in the RGB representation. 
The first layer convolves 64 $8 \times 8$ kernels with the input, using a stride of 2 and no padding. 
The second layer downsamples the output of the first layer with a $5 \times 5$ non-overlapping max pooling. 
The third layer convolves 128 $4 \times 4$ kernels with a stride of 1 and a zero-padding of size 1 on each border.  
\begin{enumerate}
\item What is the dimensionality (scalar) of the output of the last layer? 
\item Not including the biases, how many parameters are needed for the last layer?
\end{enumerate}
}
\Answer{
\begin{enumerate}
    \item $128\times 24\times 24$
    \item Number of parameters? 
\end{enumerate}

}

% (meta)
% Exercise contributed by Philippe Lacaille
% Translation by Salem Lahlou
% label: ch9

\Exercise{
\label{ex:template}
Supposons qu'on a des données de taille $3\times64\times64$. Dans ce qui suit, donnez la configuration d'une couche d'un réseau neuronal convolutif qui satisfait les hypothèses spécifiées. Répondre avec la taille du noyau ($k$), le pas ($s$), la marge ($p$), et la dilatation (\textit{dilation} $d$, en utilisant la convention $d = 0$ pour une convolution sans dilatation). Utilisez des fenêtres carrées seulement (par exemple, même valeur de $k$ pour la hauteur et la largeur).

\begin{enumerate}
\item La taille de la sortie de la première couche est $(64,32,32)$. 
\begin{enumerate}
    \item Supposons que $k=8$ sans dilatation.
    \item Supposons que $d=6$, et que $s=2 y$.
\end{enumerate}
\item La taille de la sortie de la deuxième couche est $(64,8,8)$. 
Supposons que $p=0$ et que $d=0$.
\begin{enumerate}
    \item Spécifier $k$ et $s$ pour une couche POOL sans chevauchement.
    \item Quel serait la taille de la sortie si on avait $k=8$ et $s=4$ plutôt?
\end{enumerate} 
\item La taille de la sortie de la dernière couche est $(128,4,4)$. 
\begin{enumerate}
    \item Supposons qu'on n'utilise ni marge ni dilatation.
    \item Supposons que $d=1$, et que $p=2$.
    \item Supposons que $p=1$, et que $d=0$. 
\end{enumerate}

\end{enumerate}
}

\Answer{
${}$%placeholder
}








\end{document}