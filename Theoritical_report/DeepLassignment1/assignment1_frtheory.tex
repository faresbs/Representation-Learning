\documentclass[12pt]{article}
\usepackage[canadien]{babel} 
\usepackage[utf8]{inputenc}
\usepackage[T1]{fontenc}
\usepackage{fancyhdr}
\usepackage{graphicx}
\usepackage{enumerate}
\usepackage{amsmath}
\usepackage{amssymb}
\usepackage{subfigure}
\usepackage{amsmath}
\usepackage{amssymb}
\usepackage{bm}
\usepackage{url}
\usepackage{todonotes}

% If your paper is accepted, change the options for the package
% aistats2e as follows:
%
%\usepackage[accepted]{aistats2e}
%
% This option will print headings for the title of your paper and
% headings for the authors names, plus a copyright note at the end of
% the first column of the first page.
\setlength{\parindent}{0cm}
\addtolength{\oddsidemargin}{-2cm}
\addtolength{\evensidemargin}{-2cm}
\setlength{\textwidth}{17.78cm}
\addtolength{\topmargin}{-2.25cm}
\setlength{\textheight}{24.24cm}
\addtolength{\parskip}{5mm}
\pagestyle{fancy}

%************
%* COMMANDS *
%************

\input{math_commands.tex}

\newif\ifexercise
\exercisetrue
%\exercisefalse

\newif\ifsolution
\solutiontrue
%\solutionfalse

\usepackage{booktabs}
\usepackage[chapter]{algorithm}
\usepackage{algorithmic}
% Include chapter number in algorithm number
\renewcommand{\thealgorithm}{\arabic{chapter}.\arabic{algorithm}}

\usepackage{amsthm}
\theoremstyle{definition}
\newtheorem{exercise}{Question}%[chapter]
\newtheorem{answer}{Answer} % asterisk to remove ordering
\newcommand{\Exercise}[1]{
\begin{exercise}
\ifexercise#1\fi
\end{exercise}
}
\newcommand{\Answer}[1]{
\ifsolution\begin{answer}#1\end{answer}\fi
}

\newif\ifexercise
\exercisetrue
%\exercisefalse

\newif\ifsolution
\solutiontrue
%\solutionfalse

\usepackage{enumitem}
\newcommand{\staritem}{
\addtocounter{enumi}{1}
\item[$\phantom{x}^{*}$\theenumi]}
\setlist[enumerate,1]{leftmargin=*, label=\arabic*.}

\begin{document}


\fancyhead{}
\fancyfoot{}

\fancyhead[L]{
  \begin{tabular}[b]{l}
    IFT6135-H2019  \\
    Prof: Aaron Courville \\
  \end{tabular}
}
\fancyhead[R]{
  \begin{tabular}[b]{r}
    Devoir 1, Partie théorique  \\
    Multilayer Perceptrons and Convolutional Neural Networks \\
  \end{tabular}
}

\vspace{1cm}

{\bf A rendre le: 16 février 2019}\\

\vspace{-0.5cm}
\underline{Instructions}
%Laissez des traces de votre démarche pour toutes les questions! \\
\renewcommand{\labelitemi}{\textbullet}
\begin{itemize}
\item \emph{Montrez votre démarche pour toutes les questions!}
\item \emph{Utilisez un logiciel de traitement de texte comme LaTeX.}
\item \emph{Vous devez soumettre toutes vos réponses sur la page Gradescope du cours.}
\end{itemize}

% (meta) 
% Exercise contributed by Louis-Guillaume Gagnon
% label: ch6

\Exercise{
\label{ex:heaviside}
Using the following definition of the derivative and the definition of the Heaviside step function:
$$
\frac{d}{dx}f(x) = 
\lim_{\epsilon\rightarrow 0}\ \frac{f(x + \epsilon) - f(x)}{\epsilon}
\qquad\quad
H(x)=\begin{cases}
1 & \text{if $x > 0$}\\
\frac{1}{2} & \text{if $x = 0$}\\
0 & \text{if $x < 0$}
\end{cases}
$$

\begin{enumerate}
\item Show that the derivative of the rectified linear unit $g(x) = \max\{0, x\}$, \textbf{wherever it exists}, is equal to the Heaviside step function.

\item Give two alternative definitions of $g(x)$ using $H(x)$.

\item Show that $H(x)$ can be well approximated by the sigmoid function $\sigma(x)=\frac{1}{1+e^{-k{x}}}$ asymptotically (i.e for large $k$), where $k$ is a parameter. 
%\item Show that the derivative of $H(x)$ is equal to $$\delta(x)=\begin{cases}
%\infty & \textnormal{if $x = 0$}\\
%0 & \textnormal{if $x \neq 0$.}
%\end{cases}$$

\staritem Although the Heaviside step function is not differentiable, we can define its {\bf distributional derivative}. 
For a function $F$, consider the functional $F[\phi] = \int_{\R}F(x)\phi(x)d x$, where $\phi$ is a smooth function (infinitely differentiable) with compact support ($\phi(x)= 0$ whenever  $|x| \geq A$, for some $A>0$). 

Show that whenever $F$ is differentiable, 
$F'[\phi] = - \int_{\R}F(x)\phi'(x)d x$.  
Using this formula as a definition in the case of non-differentiable functions, show that $H'[\phi]=\phi(0)$. 
($\delta[\phi]\doteq\phi(0)$ is known as the Dirac delta function.)
\end{enumerate}
}

\Answer{Q1.1:
$$
\lim_{\epsilon\rightarrow 0}\ \frac{g(x + \epsilon) - g(x)}{\epsilon}
=\lim_{\epsilon\rightarrow 0}\ \frac{max(0,x + \epsilon) - max(0,x)}{\epsilon}
=\lim_{\epsilon\rightarrow 0}\ \frac{x + \epsilon - x}{\epsilon}
=1 \;\;\;\text{for $x > 0$}
$$
As seen below, the right and left limits of the rectified linear unit are not equal, hence it is no differentiable at zero. This is due to the fact that both $x+\epsilon$ and $x$ are negative on the left side of zero but positive on the right side.
$$
\lim_{\epsilon\rightarrow 0^+}\ \frac{max(0,x + \epsilon) - max(0,x)}{\epsilon}=
\lim_{\epsilon\rightarrow 0^+}\ \frac{x + \epsilon - x}{\epsilon}
=1
$$
$$
\lim_{\epsilon\rightarrow 0^-}\ \frac{max(0,x + \epsilon) - max(0,x)}{\epsilon}=
\lim_{\epsilon\rightarrow 0
^-}\ \frac{0 - 0}{\epsilon}
=0& 
$$
Similarly, since $x+\epsilon$ and $x$ are negative on the left side of zero, we find that the derivative of the rectified linear unit is zero.
$$
\lim_{\epsilon\rightarrow 0}\ \frac{g(x + \epsilon) - g(x)}{\epsilon}
=\lim_{\epsilon\rightarrow 0}\ \frac{max(0,x + \epsilon) - max(0,x)}{\epsilon}
=\lim_{\epsilon\rightarrow 0}\ \frac{0-0}{\epsilon}
=0\;\;\text{for $x < 0$}
$$
In conclusion:\\
$$
\frac{d}{dx}g(x)=H(x)\;\text{ for $x \neq 0$}
$$
Q 1.2: Two alternative definitions of Relu could be: $g(x)=xH(x)$ and $g(x)=\int_0^x H(x) dx$.\\ 
Q 1.3: There are two asymptotic cases:\\
$$
\lim_{k\rightarrow +\infty}\sigma(x)=\lim_{k\rightarrow +\infty}\frac{1}{1+e^{-k{x}}}=\begin{cases}
\lim_{k\rightarrow +\infty}\frac{1}{1+e^{-\infty}}=1 & \text{if $x > 0$}\\
\lim_{k\rightarrow+\infty}\frac{1}{1+e^{\infty}}=0 & \text{if $x < 0$}
\end{cases}
$$
$$
\lim_{k\rightarrow -\infty}\sigma(x)=\lim_{k\rightarrow +\infty}\frac{1}{1+e^{-k{x}}}=\begin{cases}
\lim_{k\rightarrow -\infty}\frac{1}{1+e^{+\infty}}=0 & \text{if $x > 0$}\\
\lim_{k\rightarrow-\infty}\frac{1}{1+e^{-\infty}}=1 & \text{if $x < 0$}
\end{cases}
$$
In conclusion, when $k\rightarrow+\infty$, $\sigma(x)$ exactly mimics the behavior of the Heaviside function $H(x)$. \\
Q 1.4: Since the derivative of a function is another function, we define: $F'[x] =G(x)$. Using the definition of the functional/distributional derivative, we have:
$$G[\phi] =  \int_{\R}G(x)\phi(x)d x \Rightarrow F'[\phi]=   \int_{\R}F'(x)\phi(x)d x$$
Using integration by parts $\int udv=uv-\int vdu$ with $u=\phi(x)$ and $dv=F'(x)dx$, we have:\\
$$
F'[\phi]=   \int_{\R}F'(x)\phi(x)d x=F(x)\phi(x)|_{\R}-\int_{\R}F(x)\phi'(x)d x
=-\int_{\R}F(x)\phi'(x)d x
$$

$$
F(x)\phi(x)|_{\R}=F(x)\phi(x)|^{+\infty}_{-\infty}=F(x)\phi(x)|^{A}_{-A}=F(A)\phi(A)-F(-A)\phi(-A)=0
$$
Here we have used the fact that $\phi (x)$ has a compact support, as defined in the problem i.e that it $\phi(x)=0$ on the interval $|x|\geq A$ for some positive $A$.  
}
% (meta) 
% Exercise contributed by Louis-Guillaume Gagnon
% label: ch6

\Exercise{
\label{ex:grad_softmax}
Let $x$ be an $n$-dimensional vector. 
Recall the softmax function: $S: \vx \in \R^n \mapsto S(\vx) \in \R^n$ such that $S(\vx)_i=\frac{e^{\vx_i}}{\sum_j e^{\vx_j}}$; 
the diagonal function: $\text{diag}(\vx)_{ij}=\vx_i$ if $i=j$ and $\text{diag}(\vx)_{ij}=0$ if $i\neq j$;
and the Kronecker delta function: $\delta_{ij}=1$ if $i=j$ and $\delta_{ij}=0$ if $i\neq j$.  

\begin{enumerate}
\item Show that the derivative of the softmax function is $\frac{d S(\vx)_i}{d \vx_j}=S(\vx)_i\left(\delta_{ij}-S(\vx)_j\right)$.

\item Express the Jacobian matrix $\frac{\partial S(\vx)}{\partial \vx}$ using matrix-vector notation. 
Use $\text{diag}(\cdot)$.

\item Compute the Jacobian of the sigmoid function $\sigma(\vx) = 1/(1 + e^{-\vx})$.

\item Let $\vy$ and $\vx$ be $n-$dimensional vectors related by $\vy = f(\vx)$, $L$ be an unspecified differentiable loss function.
According to the chain rule of calculus, $ \nabla_\vx L = (\frac{\partial \vy}{\partial \vx})^\top  \nabla_\vy L$, which takes up $\gO(n^2)$ computational time in general. 
Show that if $f(\vx)=\sigma(\vx)$ or $f(\vx)=S(\vx)$, the above matrix-vector multiplication can be simplified to a $\gO(n)$ operation.
\end{enumerate}
}

\Answer{
${}$%placeholder
Write your answer here.
}
% (meta) 
% Exercise contributed by Antoine Lefebvre-Brossard
% Translation by Salem Lahlou
% label: ch6

\Exercise{
\label{ex:properties_softmax}
On rappelle la définition de la fonction softmax: $S(\vx)_i={e^{\vx_i}}/{\sum_j e^{\vx_j}}$.
\begin{enumerate}
\item Montrez que la fonction softmax est invariante aux translations, c'est-à-dire : $S(\vx+c) = S(\vx)$, où $c$ est une constante.

\item Montrez que la fonction softmax n'est pas invariante aux multiplications scalaires.
On définit $S_c(\vx)=S(c\vx)$ où $c\geq0$. Quels seraient les effets si on choisissait $c = 0$ ou $c \rightarrow \infty$. 

\item Soit $\vx \in \R^2$ un vecteur. On peut représenter une probabilité catégorique sur deux classes en utilisant la fonction softmax. Montrez que $S(x)$ peut être reparamétrisée en utilisant la fonction sigmoïde, c'est-à-dire : $S(\vx)=[\sigma(z), 1-\sigma(z)]^\top$ où $z$ est un scalaire, qu'il faut exprimer en fonction de $\vx$.

\item Soit $\vx \in \R^K$ un vecteur ($K\geq2$).
Montrez que $S(\vx)$ peut être représentée avec $K-1$ paramètres, c.-à-d.
$S(\vx)=S([0, y_1, y_2, ..., y_{K-1}]^\top)$ où $y_{i}$ sont des scalaires, à exprimer en fonction de $\vx$ pour $i\in\{1,...,K-1\}$. 
\end{enumerate}
}

\Answer{
${}$%placeholder
}
% (meta)
% Exercise contributed by Julian Zaidi
% Translation by Salem Lahlou
% label: ch6

\Exercise{
\label{ex:activation_transform}
On considère un réseau de neurones à deux couches $y:\R^D\rightarrow\R^K$ de la forme suivante:
$$
y(x, \Theta, \sigma)_k=\sum_{j=1}^M \omega_{kj}^{(2)}\sigma\left(\sum_{i=1}^D \omega_{ji}^{(1)}x_i+\omega_{j0}^{(1)}\right)+\omega_{k0}^{(2)}
$$
pour $1\leq k\leq K$. Les paramètres du réseau sont $\Theta=(\omega^{(1)},\omega^{(2)})$. La fonction d'activation utilisée est $\sigma$, la fonction logistique. 
Montrez qu'il existe un réseau équivalent de la même forme, avec des paramètres  $\Theta'=(\tilde{\omega}^{(1)},\tilde{\omega}^{(2)})$, avec la fonction d'activation $\tanh$, tel que $y(x, \Theta', \tanh)=y(x, \Theta, \sigma)$ pour tout $x \in \R^D$. Exprimez $\Theta'$ en fonction de $\Theta$.
}

\Answer{
${}$%placeholder
}

% (meta)
% Exercise contributed by Aristide Baratin
% Translation by Salem Lahlou
% label: ch6

\Exercise{
\label{ex:finite_sample_uat}
Soit $N$ un entier strictement positif. On veut montrer que pour toute fonction $f:\R^n \to \R^m$ et pour tout échantillon $\gS\subset \R^n$ de taille $N$, il existe un ensemble de paramètres pour un réseau de neurones à deux couches, tel que la sortie $y(\vx)$ correspond à $f(\vx)$ pour tout $\vx \in \gS$. En d'autres termes, on veut interpoler la fonction $f$ avec le réseau de neurones $y$ pour tout ensemble fini $\gS$.

\begin{enumerate}
\item Écrivez la forme générique de la fonction $y: \R^n \to \R^m$ qui définit un réseau de neurones à deux couches avec $N-1$ neurones dans la couche cachée (\textit{hidden units}), avec une fonction d'activation $\phi$, et une fonction linéaire à la dernière couche (output linéaire), en fonction des poids et biais $(\mW^{(1)}, \vb^{(1)})$ et $(\mW^{(2)}, \vb^{(2)})$.
\item Dans le reste de cet exercice, on se restreint au cas où $\mW^{(1)} = [\vw, \cdots, \vw]^T$ pour un certain $\vw \in \R^n$ (c'est-à-dire que les lignes de la matrice $\mW^{(1)}$ sont toutes pareilles).
Montrez que le problème d'interpolation sur l'ensemble $\gS=\{\vx^{(1)}, \cdots \vx^{(N)}\} \subset \R^n$ peut être réduit à la résolution d'une équation matricielle $\mM\tilde{\mW}^{(2)}=\mF$, $\tilde{\mW}^{(2)}$ et $\mF$ étant deux matrices de taille $N \times m$ définies par 
$$\tilde{\mW}^{(2)}=[\mW^{(2)}, \vb^{(2)}]^\top \qquad\qquad \mF=[f(\vx^{(1)}), \cdots, f(\vx^{(N)})]^\top$$
Exprimez la matrice $\mM$ de taille $N \times N$ en fonction de $\vw$, $\vb^{(1)}$, $\phi$ et $\vx^{(i)}$.
\staritem {\bf Preuve avec la fonction d'activation ReLU.} Supposons que les  $\vx^{(i)}$ sont tous distincts. On choisit $\vw$ tel que $\vw^\top \vx^{(i)}$ sont aussi tous distincts (Essayez de montrer l'existence d'un tel $\vw$, mais ce n'est pas requis pour le devoir - voir Assignment 0). On définit $\vb^{(1)}_j = -\vw^\top \vx^{(j)} + \epsilon$, où $\epsilon >0$. Trouvez une valeur de $\epsilon$ telle que $\mM$ est une matrice triangulaire à éléments diagonaux non-nuls. Conclure. (Indice: Définir un ordre sur les $\vw^\top\vx^{(i)}$)
\staritem {\bf Preuve avec des fonctions d'activation similaires à la sigmoïde.} Supposons que $\phi$ est continue, bornée, $\phi(-\infty)=0$ et $\phi(0)>0$. 
On écrit $\vw$ comme $\vw=\lambda\vu$. On définit $\vb^{(1)}_j = -\lambda \vu^\top \vx^{(j)}$. En laissant $\vu$ fixe, montrez que $\lim_{\lambda\to +\infty} {\mM}$ est une matrice triangulaire à éléments diagonaux non-nuls. Conclure. (A noter que cela préserve le fait que les $\vw^\top \vx^{(i)}$ sont distincts.)
\end{enumerate}
}

\Answer{
${}$%placeholder
}

% (meta)
% Exercise contributed by Philip Paquette
% label: ch9

\Exercise{
\label{ex:1dconv_eval}
Compute the \textit{full}, \textit{valid}, and \textit{same} convolution (with kernel flipping) for the following 1D matrices: $\begin{bmatrix}1,2,3,4\end{bmatrix} * \begin{bmatrix}1,0,2\end{bmatrix}$
}

\Answer{
Assuming that the given kernel is not already flipped: Flipped kernel: $[2,0,1]$\\
All convolutions are done with stride: 1\\
Full convolution (with double zero padding): $[0,0,1,2,3,4,0,0]*[2,0,1]=[1,2,5,6,6,8]$\\
Same convolution (with zero padding): $[0,1,2,3,4,0]*[2,0,1]=[2,5,8,6]$\\
Valid convolution:$[1,2,3,4]*[2,0,1]=[5,8]$
}
% (meta)
% Exercise contributed by Mariane Maynard
% Translation by Salem Lahlou
% label: ch9

\Exercise{
\label{ex:convolution_shape}
On considère un réseau de neurones à convolution. On suppose que l'entrée (\textit{input}) est une image en couleurs de taille $256 \times 256$ dans la représentation Rouge Vert Bleu (\textit{RGB}). La première couche convolue 64 noyaux $8 \times 8$ avec l'entrée, en utilisant un pas (\textit{stride}) de 2, et une marge (\textit{padding}) nulle de zéro. La deuxième couche sous-échantillonne (\textit{downsampling}) la sortie (\textit{output}) de la première couche avec un \textit{max-pool} $5 \times 5$ sans chevauchement (\textit{no overlapping}). La troisième couche convolue 128 noyaux $4 \times 4$ avec un pas de $1$, et une marge de $1$ de chaque côté.
 
\begin{enumerate}
\item Quelle est la dimension de la sortie à la dernière couche? 
\item Sans compter les biais, combien de paramètres sont requis pour la dernière couche?
\end{enumerate}
}

\Answer{
${}$%placeholder
}

% (meta)
% Exercise contributed by Philippe Lacaille
% Translation by Salem Lahlou
% label: ch9

\Exercise{
\label{ex:template}
Supposons qu'on a des données de taille $3\times64\times64$. Dans ce qui suit, donnez la configuration d'une couche d'un réseau neuronal convolutif qui satisfait les hypothèses spécifiées. Répondre avec la taille du noyau ($k$), le pas ($s$), la marge ($p$), et la dilatation (\textit{dilation} $d$, en utilisant la convention $d = 0$ pour une convolution sans dilatation). Utilisez des fenêtres carrées seulement (par exemple, même valeur de $k$ pour la hauteur et la largeur).

\begin{enumerate}
\item La taille de la sortie de la première couche est $(64,32,32)$. 
\begin{enumerate}
    \item Supposons que $k=8$ sans dilatation.
    \item Supposons que $d=6$, et que $s=2 y$.
\end{enumerate}
\item La taille de la sortie de la deuxième couche est $(64,8,8)$. 
Supposons que $p=0$ et que $d=0$.
\begin{enumerate}
    \item Spécifier $k$ et $s$ pour une couche POOL sans chevauchement.
    \item Quel serait la taille de la sortie si on avait $k=8$ et $s=4$ plutôt?
\end{enumerate} 
\item La taille de la sortie de la dernière couche est $(128,4,4)$. 
\begin{enumerate}
    \item Supposons qu'on n'utilise ni marge ni dilatation.
    \item Supposons que $d=1$, et que $p=2$.
    \item Supposons que $p=1$, et que $d=0$. 
\end{enumerate}

\end{enumerate}
}

\Answer{
${}$%placeholder
}








\end{document}