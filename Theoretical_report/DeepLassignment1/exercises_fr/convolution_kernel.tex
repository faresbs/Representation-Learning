% (meta)
% Exercise contributed by Philippe Lacaille
% Translation by Salem Lahlou
% label: ch9

\Exercise{
\label{ex:template}
Supposons qu'on a des données de taille $3\times64\times64$. Dans ce qui suit, donnez la configuration d'une couche d'un réseau neuronal convolutif qui satisfait les hypothèses spécifiées. Répondre avec la taille du noyau ($k$), le pas ($s$), la marge ($p$), et la dilatation (\textit{dilation} $d$, en utilisant la convention $d = 0$ pour une convolution sans dilatation). Utilisez des fenêtres carrées seulement (par exemple, même valeur de $k$ pour la hauteur et la largeur).

\begin{enumerate}
\item La taille de la sortie de la première couche est $(64,32,32)$. 
\begin{enumerate}
    \item Supposons que $k=8$ sans dilatation.
    \item Supposons que $d=6$, et que $s=2 y$.
\end{enumerate}
\item La taille de la sortie de la deuxième couche est $(64,8,8)$. 
Supposons que $p=0$ et que $d=0$.
\begin{enumerate}
    \item Spécifier $k$ et $s$ pour une couche POOL sans chevauchement.
    \item Quel serait la taille de la sortie si on avait $k=8$ et $s=4$ plutôt?
\end{enumerate} 
\item La taille de la sortie de la dernière couche est $(128,4,4)$. 
\begin{enumerate}
    \item Supposons qu'on n'utilise ni marge ni dilatation.
    \item Supposons que $d=1$, et que $p=2$.
    \item Supposons que $p=1$, et que $d=0$. 
\end{enumerate}

\end{enumerate}
}

\Answer{
${}$%placeholder
}
