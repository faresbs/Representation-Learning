% (meta) 
% Exercise contributed by Louis-Guillaume Gagnon
% Translation by Salem Lahlou
% label: ch6

\Exercise{
\label{ex:heaviside}
En utilisant les définitions de la dérivée et de la fonction de \textit{Heaviside} (fonction marche d'escalier) suivantes:
$$
\frac{d}{dx}f(x) = 
\lim_{\epsilon\rightarrow 0}\ \frac{f(x + \epsilon) - f(x)}{\epsilon}
\qquad\quad
H(x)=\begin{cases}
1 & \text{if $x > 0$}\\
\frac{1}{2} & \text{if $x = 0$}\\
0 & \text{if $x < 0$}
\end{cases}
$$

\begin{enumerate}
\item Montrez que la dérivée de la fonction d'activation ReLU (Unité de Rectification Linéaire) $g(x) = \max\{0, x\}$, \textbf{partout où elle existe} est égale a la fonction de Heaviside.


\item Donnez deux définitions alternatives de $g(x)$ en utilisant $H(x)$.

\item Montrez qu'on peut bien approximer $H(x)$ en utilisant la fonction logistique (la sigmoïde) $\sigma(x)=\frac{1}{1+e^{-k{x}}}$ asymptotiquement (c.-à-d. pour des valeurs de $k$ large), $k$ étant un paramètre. 
%\item Show that the derivative of $H(x)$ is equal to $$\delta(x)=\begin{cases}
%\infty & \textnormal{if $x = 0$}\\
%0 & \textnormal{if $x \neq 0$.}
%\end{cases}$$

\staritem Même si la fonction de Heaviside est non-dérivable, on peut définir sa {\bf dérivée distributionelle}. 
Pour une fonction $F$, considérez la fonctionnelle $F[\phi] = \int_{\R}F(x)\phi(x)d x$,  $\phi$ étant une fonction lisse (indéfiniment dérivable, c.-à-d. dans $\mathcal{C}^\infty$) à support compact ($\phi(x)= 0$ quand  $|x| \geq A$, pour un certain $A>0$). 

Montrez que si $F$ est dérivable, alors
$F'[\phi] = - \int_{\R}F(x)\phi'(x)d x$. En utilisant cette formule comme une définition de la dérivée distributionelle dans le cas des fonctions non-dérivables, montrez que
 $H'[\phi]=\phi(0)$. 
($\delta[\phi]\doteq\phi(0)$ est la fonction delta de Dirac (ou distribution de Dirac))
\end{enumerate}
}

\Answer{
${}$%placeholder
}