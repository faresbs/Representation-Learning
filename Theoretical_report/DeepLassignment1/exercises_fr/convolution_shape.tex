% (meta)
% Exercise contributed by Mariane Maynard
% Translation by Salem Lahlou
% label: ch9

\Exercise{
\label{ex:convolution_shape}
On considère un réseau de neurones à convolution. On suppose que l'entrée (\textit{input}) est une image en couleurs de taille $256 \times 256$ dans la représentation Rouge Vert Bleu (\textit{RGB}). La première couche convolue 64 noyaux $8 \times 8$ avec l'entrée, en utilisant un pas (\textit{stride}) de 2, et une marge (\textit{padding}) nulle de zéro. La deuxième couche sous-échantillonne (\textit{downsampling}) la sortie (\textit{output}) de la première couche avec un \textit{max-pool} $5 \times 5$ sans chevauchement (\textit{no overlapping}). La troisième couche convolue 128 noyaux $4 \times 4$ avec un pas de $1$, et une marge de $1$ de chaque côté.
 
\begin{enumerate}
\item Quelle est la dimension de la sortie à la dernière couche? 
\item Sans compter les biais, combien de paramètres sont requis pour la dernière couche?
\end{enumerate}
}

\Answer{
${}$%placeholder
}
